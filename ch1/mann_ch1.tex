\documentclass[11pt]{article}
\usepackage{amssymb}
\usepackage{amsthm}
\usepackage{enumitem}
\usepackage{amsmath}
\usepackage{bm}
\usepackage{adjustbox}
\usepackage{mathrsfs}
\usepackage{graphicx}
\usepackage{siunitx}
\usepackage[mathscr]{euscript}

\title{\textbf{Solved selected problems of Lagrangian and Hamiltonian by Mann}}
\author{Franco Zacco}
\date{}

\addtolength{\topmargin}{-3cm}
\addtolength{\textheight}{3cm}

\newcommand{\R}{\mathbb{R}}
\newcommand{\C}{\mathbb{C}}
\newcommand{\hatr}{\bm{\hat{r}}}
\newcommand{\hatx}{\bm{\hat{x}}}
\newcommand{\haty}{\bm{\hat{y}}}
\newcommand{\hatz}{\bm{\hat{z}}}
\newcommand{\hatth}{\bm{\hat{\theta}}}
\newcommand{\hatphi}{\bm{\hat{\phi}}}
\newcommand{\hatrho}{\bm{\hat{\rho}}}
\newcommand{\ei}[1]{\vec{\bm{e}}_#1}
\theoremstyle{definition}
\newtheorem*{solution*}{Solution}
\renewcommand*{\proofname}{\textbf{Solution}}

\begin{document}
\maketitle
\thispagestyle{empty}

\begin{proof}{\textbf{1.1}}
Let us integrate the acceleration $2c$ to obtain the velocity $\dot{r}(t)$
as follows
\begin{align*}
    \dot{r}(t) = \int 2c~dt = 2ct + C
\end{align*}
But at $t = 0$ we know that $\dot{r}(t) = b$ then $C = b$ and hence
\begin{align*}
    \dot{r}(t) = 2ct + b
\end{align*}
Now, we integrate again to obtain the position $r(t)$ as follows
\begin{align*}
    r(t) = \int 2ct + b~dt = ct^2 + bt + C
\end{align*}
But at $t = 0$ we know that $r(t) = a$ then $C = a$ and hence
\begin{align*}
    r(t) = ct^2 + bt + a
\end{align*}


\end{proof}

\end{document}