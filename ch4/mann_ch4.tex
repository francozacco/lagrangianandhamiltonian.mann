\documentclass[11pt]{article}
\usepackage{amssymb}
\usepackage{amsthm}
\usepackage{enumitem}
\usepackage{amsmath}
\usepackage{bm}
\usepackage{adjustbox}
\usepackage{mathrsfs}
\usepackage{graphicx}
\usepackage{siunitx}
\usepackage{physics}
\usepackage[mathscr]{euscript}

\title{\textbf{Solved selected problems of Lagrangian and Hamiltonian by Mann}}
\author{Franco Zacco}
\date{}

\addtolength{\topmargin}{-3cm}
\addtolength{\textheight}{3cm}

\newcommand{\R}{\mathbb{R}}
\newcommand{\C}{\mathbb{C}}
\newcommand{\hatr}{\bm{\hat{r}}}
\newcommand{\hatx}{\bm{\hat{x}}}
\newcommand{\haty}{\bm{\hat{y}}}
\newcommand{\hatz}{\bm{\hat{z}}}
\newcommand{\hatth}{\bm{\hat{\theta}}}
\newcommand{\hatphi}{\bm{\hat{\phi}}}
\newcommand{\hatrho}{\bm{\hat{\rho}}}
\newcommand{\ei}[1]{\vec{\bm{e}}_#1}
\theoremstyle{definition}
\newtheorem*{solution*}{Solution}
\renewcommand*{\proofname}{\textbf{Solution}}

\begin{document}
\maketitle
\thispagestyle{empty}

\begin{proof}{\textbf{4.2}}
Let
\begin{align*}
    x(t) = a\cos(\omega t + \phi) \qquad
    \dot{x}(t) = - a\omega \sin(\omega t + \phi)
\end{align*}
To compute the average kinetic energy over a cycle let us assume that $\phi = 0$
or what is the same that we measure the time of the cycle from $\phi$ to
$\omega T + \phi$, hence
\begin{align*}
    \langle\frac{1}{2}m\dot{x}^2\rangle
    &= \frac{1}{T}\int_{0}^T \frac{1}{2}m a^2\omega^2 \sin^2(\omega t)~dt\\
    &= \frac{ma^2\omega^2}{2T}\int_{0}^T \sin^2(\omega t)~dt\\
    &= \frac{ma^2\omega^2}{2T}\bigg[\frac{t}{2}-\frac{\sin(2\omega t)}{4\omega}\bigg]_0^T\\
    &= \frac{ma^2\omega^2}{2T}\bigg[\frac{T}{2}-\frac{\sin(2\omega T)}{4\omega}\bigg]\\
    &= \frac{\pi^2ma^2}{T}
\end{align*}
Where we used that $\omega = 2\pi/T$ and hence $\sin(4\pi) = 0$.\\
The average potential energy over a cycle is given by
\begin{align*}
    \langle\frac{1}{2}kx^2\rangle
    &= \frac{1}{T}\int_{0}^T \frac{1}{2}k a^2 \cos^2(\omega t)~dt\\
    &= \frac{ka^2}{2T}\int_{0}^T \cos^2(\omega t)~dt\\
    &= \frac{m\omega^2a^2}{2T}\bigg[\frac{t}{2} + \frac{\sin(2\omega t)}{4\omega}\bigg]_0^T\\
    &= \frac{ma^2\omega^2}{2T}\bigg[\frac{T}{2} + \frac{\sin(2\omega T)}{4\omega}\bigg]\\
    &= \frac{\pi^2ma^2}{T}
\end{align*}
Where we used that $k = m\omega^2$.\\
Finally, we compute the average energy as follows
\begin{align*}
    \langle\frac{1}{2}m\dot{x}^2 + \frac{1}{2}kx^2\rangle
    &= \frac{1}{T}\int_{0}^T \frac{1}{2}m a^2\omega^2 \sin^2(\omega t)
    + \frac{1}{2}k a^2 \sin^2(\omega t)~dt\\
    &= \frac{m\omega^2a^2}{2T}\int_{0}^T \sin^2(\omega t) + \cos^2(\omega t)~dt\\
    &= \frac{m\omega^2a^2}{2T}\bigg[
        \frac{t}{2}-\frac{\sin(2\omega t)}{4\omega}
        + \frac{t}{2} + \frac{\sin(2\omega t)}{4\omega}
    \bigg]_0^T\\
    &= \frac{m\omega^2a^2}{2T}\cdot T\\
    &= \frac{2\pi^2ma^2}{T}
\end{align*}
Therefore we see that
\begin{align*}
    \langle\frac{1}{2}m\dot{x}^2\rangle = \langle\frac{1}{2}kx^2\rangle
    = \langle\frac{1}{2}m\dot{x}^2 + \frac{1}{2}kx^2\rangle
\end{align*}
\end{proof}

\end{document}